\documentclass[a4paper, oneside, final]{article}
\usepackage[T1]{fontenc}
\usepackage[utf8]{inputenc}
\usepackage[british]{babel}
\usepackage{amsmath}
\usepackage{amsthm}
\usepackage{verbatim}

\renewcommand{\britishhyphenmins}{22} 

\let\fref\undefined
\let\Fref\undefined

\usepackage{graphicx}
\usepackage{amssymb}

\setcounter{secnumdepth}{1} % Sæt overskriftsnummereringsdybde. Disable = -1.
\setlength{\parskip}{0.25in}

\pagestyle{plain}

\title{Topics in Programming Languages}

\begin{document}

\maketitle

\section{Eden}

\section{Par Monad}

The Par Monad was originally described in the paper ``A monad for deterministic parallelism``.

\section{Implementing the Par Monad}

\section{Conclusion}

\end{document}
